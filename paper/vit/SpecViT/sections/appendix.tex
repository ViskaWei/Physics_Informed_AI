\section{Supplementary tables for baselines, scaling, and SNR bins}
\label{app:tables}

This Appendix provides expanded numerical results referenced in the main Results section, including additional metrics for \textsc{SpecViT}, the full scaling table, and per-SNR values used in Figure~\ref{fig:snr_ceiling}.

\subsection{Supplementary metrics for the primary \textsc{SpecViT} model}
\label{app:tables_metrics}

\begin{deluxetable}{lc}
\tablecaption{Supplementary metrics for \textsc{SpecViT} trained on $10^6$ spectra and evaluated on the 10k-spectrum test set.\label{tab:supp_metrics}}
\tablehead{
\colhead{Metric} & \colhead{Value}
}
\startdata
$R^2$ & 0.711 \\
MAE (dex) & 0.372 \\
RMSE $\sigma_{\mathrm{ViT}}$ (dex) & 0.64 \\
Best checkpoint epoch & 128 \\
\enddata
\end{deluxetable}

\subsection{Full scaling numbers}
\label{app:tables_scaling}

\begin{deluxetable}{rccc}
\tablecaption{Test-set $R^2$ versus training set size $N$.\label{tab:scaling_numbers}}
\tablehead{
\colhead{$N$} & \colhead{\textsc{SpecViT}} & \colhead{LightGBM} & \colhead{Ridge}
}
\startdata
$5\times10^4$  & 0.434 & 0.488 & 0.442 \\
$1\times10^5$  & 0.596 & 0.553 & 0.475 \\
$2\times10^5$  & 0.673 & 0.547 & 0.474 \\
$5\times10^5$  & 0.709 & 0.574 & 0.490 \\
$1\times10^6$  & 0.711 & 0.614 & 0.50 \\
\enddata
\tablecomments{All models are evaluated on the same 10k-spectrum test split.}
\end{deluxetable}

\subsection{Per-SNR values underlying Figure~\ref{fig:snr_ceiling}}
\label{app:tables_snr}

\begin{deluxetable*}{rccccc}
\tablecaption{Per-SNR performance and theoretical ceiling values.\label{tab:snr_numbers}}
\tablehead{
\colhead{Magnitude} &
\colhead{SNR} &
\colhead{$R^2_{\mathrm{ViT}}$} &
\colhead{$R^2_{\mathrm{LGBM}}$} &
\colhead{$R^2_{\max}$ (Fisher, 5D)} &
\colhead{$R^2_{\max}-R^2_{\mathrm{ViT}}$}
}
\startdata
20.0 & 24.0 & 0.90 & 0.87 & 0.989 & 0.09 \\
21.5 & 7.1  & 0.80 & 0.74 & 0.874 & 0.07 \\
22.0 & 4.6  & 0.68 & 0.60 & 0.698 & 0.02 \\
22.5 & 3.0  & 0.52 & 0.42 & 0.265 & \nodata \\
\enddata
\tablecomments{The gap is omitted in the lowest-SNR bin because the reported $R^2_{\max}$ corresponds to a median ceiling over the parameter space; Appendix~\ref{app:fisher_extra} discusses this interpretation.}
\end{deluxetable*}

\section{Tokenization and architectural sensitivity}
\label{app:ablation}

The main Results focus on performance, scaling, and proximity to the information ceiling. Here we provide supporting ablations that validate key architectural choices used for \textsc{SpecViT} in the main experiments, particularly the use of 16-pixel patches with a convolutional patch tokenizer.

\subsection{Patch size ablation}
\label{app:ablation_patch}

\begin{deluxetable}{lcc}
\tablecaption{Patch size sensitivity (Conv1D tokenization; ablation on a 50k-scale training regime).\label{tab:patch_ablation}}
\tablehead{
\colhead{Patch size} & \colhead{Validation $R^2$} & \colhead{Test $R^2$}
}
\startdata
16 & $0.582 \pm 0.045$ & $0.554 \pm 0.042$ \\
32 & $0.473 \pm 0.128$ & $0.449 \pm 0.125$ \\
64 & $0.534$ & $0.496$ \\
\enddata
\tablecomments{Values are reported as mean$\pm$std when multiple runs are available.}
\end{deluxetable}

\subsection{Tokenization method stability}
\label{app:ablation_tokenizer}

\begin{deluxetable*}{lcccc}
\tablecaption{Stability of two tokenization strategies in an architectural sweep.\label{tab:tokenizer_stability}}
\tablehead{
\colhead{Tokenizer} &
\colhead{Runs (total)} &
\colhead{Runs (finished)} &
\colhead{Success rate} &
\colhead{Best validation $R^2$}
}
\startdata
Conv1D (C1D) & 79 & 23 & 29\% & 0.631 \\
Sliding window (SW) & 15 & 0 & 0\% & \nodata \\
\enddata
\tablecomments{The SW configuration did not yield stable completed runs under the sweep settings used here; the main Results therefore adopt C1D tokenization.}
\end{deluxetable*}

\section{Additional diagnostics relative to the Fisher limit}
\label{app:fisher_extra}

This Appendix provides supporting details for the Fisher-information ceiling shown in Figure~\ref{fig:snr_ceiling}, and additional diagnostics comparing empirical residual scatter to CRLB envelopes.

\subsection{From Fisher information to an $R^2$ ceiling}
\label{app:fisher_extra_conversion}

For a spectral forward model with parameters $\theta=(\log g,\eta)$ and heteroscedastic noise covariance $\Sigma$, the Fisher information matrix is $I(\theta)=J^\top \Sigma^{-1}J$, where $J=\partial f/\partial\theta$ is the Jacobian. Marginalizing over nuisance parameters $\eta$ yields the Schur-complement CRLB for $\log g$,
\begin{equation}
\mathrm{CRLB}_{g,\mathrm{marg}}=\left(I_{gg}-I_{g\eta}I_{\eta\eta}^{-1}I_{\eta g}\right)^{-1}.
\end{equation}
We convert this variance lower bound into an $R^2$ ceiling via
\begin{equation}
R^2_{\max}=1-\frac{\mathrm{CRLB}_{g,\mathrm{marg}}}{\mathrm{Var}(\log g)}.
\end{equation}
This provides an SNR-conditioned upper bound for any unbiased estimator under the assumed noise model.

\subsection{Residual overlay with CRLB envelopes}
\label{app:fisher_extra_residual}

\begin{figure*}
\centering
\IfFileExists{figs/fisher_residual_overlay_real_dual_mag.png}{%
  \includegraphics[width=\textwidth]{figs/fisher_residual_overlay_real_dual_mag.png}%
}{%
  \fbox{\parbox[c][0.18\textheight][c]{0.95\textwidth}{\centering Missing figure: \texttt{fisher\_residual\_overlay\_real\_dual\_mag.png}}}%
}
\caption{
Residual diagnostics for the \textsc{SpecViT} model trained on $10^6$ spectra, evaluated on the 10k-spectrum test set.
The plot overlays empirical residual scatter with CRLB envelopes at two representative noise regimes (e.g., corresponding to mag $\approx 21.5$ and mag $\approx 22.5$), illustrating that the measured RMSE $\sigma_{\mathrm{ViT}}=0.64~\mathrm{dex}$ lies between the corresponding Fisher lower bounds ($\sigma_{\mathrm{Fisher}}\approx 0.43$ and $1.11~\mathrm{dex}$).
}
\label{fig:fisher_residual_overlay}
\end{figure*}

\section{Invariance of $R^2$ to linear label normalization}
\label{app:r2_invariance}

For completeness, we record a short proof that $R^2$ is invariant under linear transformations of the regression target, which justifies direct comparison of $R^2$ values across standard ($z$-score) label normalizations.

Let $y' = a y + b$ be any linear transformation with $a\neq 0$, and let $\hat{y}'=a\hat{y}+b$ be the correspondingly transformed predictions. The residual sum of squares transforms as
\begin{equation}
SS_{\mathrm{res}}'=\sum_i (y'_i-\hat{y}'_i)^2
=\sum_i \big(a(y_i-\hat{y}_i)\big)^2
=a^2 SS_{\mathrm{res}},
\end{equation}
and the total sum of squares transforms as
\begin{equation}
SS_{\mathrm{tot}}'=\sum_i (y'_i-\overline{y}')^2
=\sum_i \big(a(y_i-\overline{y})\big)^2
=a^2 SS_{\mathrm{tot}}.
\end{equation}
Therefore,
\begin{equation}
R^2
=1-\frac{SS_{\mathrm{res}}}{SS_{\mathrm{tot}}}
=1-\frac{SS_{\mathrm{res}}'}{SS_{\mathrm{tot}}'},
\end{equation}
showing that $R^2$ is unchanged by linear target normalization.

\section{Synthetic Data Generation Pipeline}
\label{app:data_generation}

This Appendix describes the procedure used to generate the synthetic stellar spectra dataset, including the BOSZ spectral library, parameter sampling, grid interpolation, and the PFS instrument noise model.

\subsection{BOSZ Spectral Library}
\label{app:data_bosz}

The underlying spectral templates are drawn from the BOSZ (Bohlin--Osmer--Sahnow) grid \citep{bohlin2017bosz}, computed using ATLAS9 stellar atmospheres \citep{castelli2004atlas9} with updated atomic and molecular line lists. The native resolution is $R\approx 50{,}000$, covering wavelengths from the ultraviolet through the near-infrared. For this work, we use the solar-scaled subset with metallicity [M/H] ranging from $-2.5$ to $+0.75$~dex in steps of $0.25$~dex, effective temperature $T_{\mathrm{eff}}$ from $3500$ to $12{,}000$~K in varying steps ($250$~K at cool temperatures), and surface gravity $\log g$ from $0.0$ to $5.0$~dex in steps of $0.5$~dex.

\subsection{Grid Interpolation}
\label{app:data_interpolation}

To generate spectra at arbitrary stellar parameters within the BOSZ grid coverage, we employ cubic spline interpolation separately along each parameter axis. Specifically, for a target parameter set $(T_{\mathrm{eff}}, \log g, [\mathrm{M/H}])$:
\begin{enumerate}
    \item Identify the enclosing hypercube of grid nodes.
    \item Perform 1D cubic spline interpolation along each axis sequentially (temperature, then gravity, then metallicity).
    \item Combine the interpolated fluxes to produce the final high-resolution template.
\end{enumerate}
This approach preserves spectral line shapes and ensures smooth transitions across parameter space. The interpolated templates are then convolved with the instrument line-spread function (LSF) and resampled to the detector wavelength grid.

\subsection{PFS Instrument Model}
\label{app:data_pfs}

The Subaru Prime Focus Spectrograph (PFS) medium-resolution (MR) arm is modeled using the \texttt{pfsspec} simulation pipeline. Key instrument parameters are:
\begin{itemize}
    \item \textbf{Wavelength coverage}: $710$--$885$~nm.
    \item \textbf{Spectral resolution}: $R\approx 5000$ ($\Delta\lambda \approx 1.6$~\AA).
    \item \textbf{Detector sampling}: $0.4$~\AA\ per pixel (4096 pixels total).
    \item \textbf{Line-spread function}: Gaussian with wavelength-dependent width derived from the optical model.
\end{itemize}
The convolution from native BOSZ resolution to PFS resolution is performed via a wavelength-dependent Gaussian kernel, followed by linear interpolation onto the fixed detector wavelength grid.

\subsection{Noise Model}
\label{app:data_noise}

Observational noise is simulated by modeling the full photon-counting chain. For a spectrum with $i$-band apparent magnitude $m_i$ and observing conditions (seeing, zenith angle, moon configuration), we compute:
\begin{enumerate}
    \item \textbf{Object counts}: The expected photon count $N_{\mathrm{obj},j}$ at each wavelength pixel $j$ is derived from the fluxed template, exposure time, and telescope/instrument throughput.
    \item \textbf{Sky counts}: The sky background $N_{\mathrm{sky},j}$ is computed from a PFS sky model (new-moon conditions, dark time).
    \item \textbf{Detector noise}: Read noise (approximately 3~e$^-$ per pixel per read) and dark current are added.
    \item \textbf{Total variance}: The per-pixel variance is $\sigma_j^2 = N_{\mathrm{obj},j} + N_{\mathrm{sky},j} + \sigma_{\mathrm{read}}^2$, following Poisson statistics for photon counts.
\end{enumerate}
The noisy flux is generated by adding Gaussian noise with standard deviation $\sigma_j$ to the noiseless (sky-subtracted, flux-calibrated) spectrum. The per-pixel error vector $\boldsymbol{\sigma}$ is stored alongside the flux and used during training for on-the-fly noise injection (Eq.~\ref{eq:noise_model}).

\subsection{Dataset Partitioning}
\label{app:data_splits}

The full dataset of $10^6$ spectra is generated in five independent shards of $2\times10^5$ spectra each, using distinct random seeds for stellar parameter sampling and noise realization. An additional $10^3$ spectra are generated for validation and $10^4$ for testing. All three splits use disjoint stellar parameter draws and noise seeds to ensure no information leakage between training and evaluation.

\section{Template Fitting Baseline}
\label{app:tempfit}

This Appendix describes the physics-based template fitting method used as a baseline in Table~\ref{tab:main_results}, which achieves $R^2=0.404$ for $\log g$ inference.

\subsection{Algorithm Overview}
\label{app:tempfit_algo}

Template fitting infers stellar parameters by maximizing the likelihood of the observed spectrum given a library of synthetic templates. For an observed spectrum $\boldsymbol{f}^{\mathrm{obs}}$ with per-pixel uncertainties $\boldsymbol{\sigma}$, the log-likelihood for a template $\boldsymbol{f}^{\mathrm{mod}}(\theta)$ at parameters $\theta = (T_{\mathrm{eff}}, \log g, [\mathrm{M/H}])$ is:
\begin{equation}
\ln \mathcal{L}(\theta) = -\frac{1}{2} \sum_{j=1}^{L} \frac{\bigl(f_j^{\mathrm{obs}} - A \cdot f_j^{\mathrm{mod}}(\theta)\bigr)^2}{\sigma_j^2},
\label{eq:tempfit_logL}
\end{equation}
where $A$ is a flux scaling factor (marginalized analytically) that accounts for distance, throughput, and continuum normalization. The best-fit parameters $\hat{\theta}$ are obtained by maximizing Eq.~\eqref{eq:tempfit_logL} over the BOSZ grid using spline-interpolated templates.

\subsection{Optimization Procedure}
\label{app:tempfit_optim}

The fitting proceeds in two stages:
\begin{enumerate}
    \item \textbf{Coarse grid search}: Evaluate the log-likelihood on a subsampled grid (every second node in each parameter direction) to identify a promising region.
    \item \textbf{Local refinement}: Use the Nelder--Mead simplex algorithm \citep{nelder1965simplex} starting from the coarse-grid optimum, with spline-interpolated templates evaluated at each function call.
\end{enumerate}
The analytic flux normalization factor is computed at each grid point as:
\begin{equation}
A^* = \frac{\sum_j f_j^{\mathrm{obs}} f_j^{\mathrm{mod}} / \sigma_j^2}{\sum_j (f_j^{\mathrm{mod}})^2 / \sigma_j^2},
\end{equation}
which corresponds to the weighted least-squares solution for a linear scaling.

\subsection{Error Weighting}
\label{app:tempfit_weights}

The per-pixel error vector $\boldsymbol{\sigma}$ is used as inverse-variance weights ($w_j = 1/\sigma_j^2$). This naturally down-weights wavelength regions with high noise (e.g., sky emission lines, atmospheric absorption features, low-flux edges) and emphasizes regions with high signal-to-noise. The error vector encodes primarily \emph{instrument and observing-condition} characteristics (detector response, sky brightness, exposure time) rather than information about the target spectrum itself, and therefore does not constitute data leakage.

\subsection{Implementation and Results}
\label{app:tempfit_results}

Template fitting is implemented using the \texttt{pfsspec} stellar fitting module. On a 1000-sample noisy test set, the method achieves:
\begin{itemize}
    \item $R^2(\log g) = 0.404$, MAE $= 0.66$~dex.
    \item $R^2(T_{\mathrm{eff}}) = 0.716$, MAE $= 261$~K.
    \item $R^2([\mathrm{M/H}]) = 0.845$, MAE $= 0.27$~dex.
\end{itemize}
The lower performance on $\log g$ compared to $T_{\mathrm{eff}}$ and metallicity reflects the weaker spectroscopic sensitivity of surface gravity: $\log g$ is encoded primarily in pressure-broadened line wings and subtle line-ratio diagnostics that are more easily overwhelmed by noise than the overall spectral shape (temperature) or line-depth modulation (metallicity).

\section{Fisher Information Computation Details}
\label{app:fisher_computation}

This Appendix provides implementation details for the Fisher-information ceiling presented in Section~\ref{subsec:results_fisher} and Appendix~\ref{app:fisher_extra}.

\subsection{Regular Grid Requirement}
\label{app:fisher_grid}

Reliable computation of the Fisher information matrix requires spectra evaluated on a \emph{regular parameter grid} so that finite-difference derivatives can be computed consistently. For each magnitude/SNR bin, we generate a dedicated grid with structure:
\begin{itemize}
    \item $T_{\mathrm{eff}}$: 10 nodes spanning 3750--6000~K (250~K steps).
    \item $\log g$: 9 nodes spanning 1.0--5.0~dex (0.5~dex steps).
    \item $[\mathrm{M/H}]$: 14 nodes spanning $-0.25$ to $+0.75$~dex (variable steps matching BOSZ grid).
\end{itemize}
This yields 1260 grid points per magnitude bin. For the 5D Fisher analysis, we additionally include $[\alpha/\mathrm{M}]$ and $[\mathrm{C/M}]$ (2--3 nodes each), increasing the grid to approximately $10 \times 9 \times 14 \times 3 \times 3 = 11{,}340$ points.

\subsection{Jacobian Computation via Finite Differences}
\label{app:fisher_jacobian}

At each interior grid point, we compute the spectral Jacobian $J = \partial f / \partial \theta$ using central finite differences along each parameter axis:
\begin{equation}
\frac{\partial f_j}{\partial \theta_k} \approx \frac{f_j(\theta_k + \Delta\theta_k) - f_j(\theta_k - \Delta\theta_k)}{2\Delta\theta_k},
\end{equation}
where $\Delta\theta_k$ is the grid spacing for parameter $k$. For a 3D parameter space $(T_{\mathrm{eff}}, \log g, [\mathrm{M/H}])$, this produces a $L \times 3$ Jacobian matrix at each grid point, where $L=4096$ is the number of wavelength pixels.

\paragraph{Numerical stability.}
Using a regular grid with consistent step sizes avoids the numerical instability encountered when using irregular (randomly sampled) parameter spacing. In early experiments, irregular grids produced CRLB values spanning 20 orders of magnitude due to inconsistent finite-difference denominators; the regular-grid approach reduces this range to approximately 3 orders of magnitude, consistent with physical expectations.

\subsection{Fisher Matrix and Marginalization}
\label{app:fisher_marginalize}

Given the Jacobian $J$ and the diagonal noise covariance $\Sigma = \mathrm{diag}(\sigma_1^2, \ldots, \sigma_L^2)$, the Fisher information matrix is:
\begin{equation}
I(\theta) = J^\top \Sigma^{-1} J \in \mathbb{R}^{d \times d},
\end{equation}
where $d$ is the number of parameters (3 for 3D, 5 for 5D). To obtain the CRLB for $\log g$ marginalized over nuisance parameters $\eta = (T_{\mathrm{eff}}, [\mathrm{M/H}], \ldots)$, we partition the Fisher matrix:
\begin{equation}
I = \begin{pmatrix} I_{gg} & I_{g\eta} \\ I_{\eta g} & I_{\eta\eta} \end{pmatrix},
\end{equation}
and compute the Schur complement:
\begin{equation}
\mathrm{CRLB}_{g,\mathrm{marg}} = \left( I_{gg} - I_{g\eta} I_{\eta\eta}^{-1} I_{\eta g} \right)^{-1}.
\end{equation}
The Schur decay factor $\rho = (I_{gg} - I_{g\eta} I_{\eta\eta}^{-1} I_{\eta g}) / I_{gg}$ quantifies the fraction of $\log g$ information retained after marginalizing over nuisance parameters; for our 3D (5D) analysis, $\rho \approx 0.69$ (0.58), indicating that approximately 30\% (42\%) of the raw $\log g$ information is ``absorbed'' by parameter degeneracy.

\subsection{CRLB to $R^2$ Conversion}
\label{app:fisher_r2}

At each grid point, we convert the marginalized CRLB to a theoretical $R^2$ ceiling:
\begin{equation}
R^2_{\max} = 1 - \frac{\mathrm{CRLB}_{g,\mathrm{marg}}}{\mathrm{Var}(\log g)},
\end{equation}
where $\mathrm{Var}(\log g)$ is the label variance over the training distribution. The reported ceiling values in Table~\ref{tab:snr_numbers} are the \emph{median} of $R^2_{\max}$ across all grid points at each magnitude, providing a representative summary. The 10th and 90th percentiles define the shaded confidence bands in Figure~\ref{fig:snr_ceiling}.

\subsection{5D vs 3D Ceiling Comparison}
\label{app:fisher_5d}

Including chemical abundances ($[\alpha/\mathrm{M}]$, $[\mathrm{C/M}]$) as additional nuisance parameters reduces the $\log g$ ceiling by increasing parameter degeneracy. The effect is SNR-dependent:
\begin{itemize}
    \item At high SNR ($\mathrm{mag} \leq 20$): $\Delta R^2_{\max} \lesssim 2\%$---chemical abundances are well-constrained and contribute little additional degeneracy.
    \item At low SNR ($\mathrm{mag} = 22.5$): $\Delta R^2_{\max} \approx 28\%$---noise amplifies correlations between $\log g$ and abundance diagnostics.
\end{itemize}
For this reason, the main paper reports the 5D marginalized ceiling as a conservative upper bound on achievable performance.
